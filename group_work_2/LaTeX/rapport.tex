\section{Description of the server}
\label{sec:Description of the server}
We implemented our server in Java, with a an adjustable size
of pending request. Our server receives request and if it is busy proccessing one,
coming request will be add to the queue and proccessed later.
 
\subsection{Computation performed by the server}
\label{sub:Computation performed by the server}
The compution performed by the server consist of calculating the power of a 
two-dimensional Matrix $M$ powered by a natural $p$
\subsection{Request and response format}
Each client request contains the matrix and the exposant. To achieve the 
measurement that were asked, we added some fields to the request to get 
some information such as computation time and and response time. We also
give and identifier to a request to map to corresponding response. 
\label{sub:Request and response format}

\section{Description of the measurement setup}
\label{sec:Description of the measurement setup}

\subsection{Used hardwares and softwares}
\label{sub:Used hardwares and softwares}

\subsection{Description of the load generator}
\label{sub:Description of the load generator}

\section{Measurements and modeling}
\label{sec:Measurements and modeling}

\subsection{Measurement of individual client requests}
\label{sub:Measurement of individual client requests}

\subsection{Measurement of the load generator}
\label{sub:Measurement of the load generator}

\subsection{Queueing station model}
\label{sub:Queueing station model}

\subsection{Improvement of the performance of the system}
\label{sub:Improvement of the performance of the system}

\subsection{Measurements on a multi-threaded server}
\label{sub:Measurements on a multi-threaded server}

\subsubsection{Bottlenecks}
\label{subs:Bottlenecks}


\subsubsection{Multi-threading and queuing station model}
\label{subs:Multi-threading and queuing station model}
