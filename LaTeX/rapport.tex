\todo[inline]{Le rapport peut contenir entre 2 pages et 2 pages et demi, A4
11pt}
\section{Introduction}

\section{Warm-up phase}

\todo[inline]{Expliquer la bonne valeur pour la warm-up phase}
\missingfigure{plot du hit rate en fonction du temps, voir
\url{http://blog.coelho.net/database/2013/08/14/postgresql-warmup/}}

\section{Task1.1}

\missingfigure[inline]{Un graphique avec 2 courbes : une pour le LFU et une
pour le LRU, avec en y le hit rate et en x la taille du cache. Attention à
choisir une valeur pour X suffisament grande par rapport aux tailles de
cache testées ; indiquer sa valeur}

\todo[inline]{Une description de l'implémentation du LFU : data structures
; checking/adding/removing operations sont-elles efficaces ?}


\section{Task2.2}

\missingfigure[inline]{Un graphique avec 6 courbes (byte hit rate et hit
rate) : 2 pour le LFU, 2 pour le LRU et 2 pour la méthode supplémentaire
choisie, avec en y le hit rate et en x la taille du cache. Attention à
choisir une valeur pour X suffisament grande par rapport aux tailles de
cache testées ; indiquer sa valeur}

\todo[inline]{Une description de la stratégie supplémentaire size-based :
data structures, ...}

\todo[inline]{Discuter des performances des 3 stratégies : comment
sont-elles différentes et quelles sont les raisons de ces différences ?}
